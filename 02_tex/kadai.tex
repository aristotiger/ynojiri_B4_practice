\documentclass[twocolumn,a4j]{jarticle}
\usepackage[dvipdfmx]{graphicx}
\usepackage{url}
\usepackage{ascmac}
\usepackage{amsmath}
\usepackage{fancyhdr}
\usepackage{color}
\usepackage{footnote}

%色付き
\newcommand{\red}[1]{{\color{red}#1}}

% 所属と氏名を右寄せにする設定
\makeatletter
  \def\@maketitle{%
  \newpage\null
  %\vskip 2em%
  \begin{center}%
  \let\footnote\thanks
    {\LARGE \@title \par}%
    %\vskip 1.5em%
  \end{center}% 追加
    \mbox{}\hfill%% 追加
    {\large
      %\lineskip .5em%
      \begin{tabular}[t]{c}%
        \@author
      \end{tabular}\par}%
    \vskip 1em%
  \begin{center}% 追加
    {\large \@date}%
  \end{center}%
  %\par\vskip 1.5em
}

% section前後の行間を詰める設定
\renewcommand{\section}{\@startsection{section}{1}{\z@}%
   {.3\Cvs \@plus.5\Cvs \@minus.2\Cvs}% % section上部の空白量
   {.2\Cvs \@plus.3\Cvs}%                % section下部の空白量
   {\reset@font\Large\bfseries}}         % フォント,フォントサイズを変更
\renewcommand{\subsection}{\@startsection{subsection}{2}{\z@}%
   {.3\Cvs \@plus.5\Cvs \@minus.2\Cvs}%
   {.2\Cvs \@plus.3\Cvs}%
   {\reset@font\large\bfseries}}
\renewcommand{\subsubsection}{\@startsection{subsubsection}{3}{\z@}%
   {.3\Cvs \@plus.5\Cvs \@minus.2\Cvs}% 
   {.2\Cvs \@plus.3\Cvs}%
   {\reset@font\normalsize\bfseries}}

\makeatother


% 余白の設定
\usepackage[top=20truemm,bottom=20truemm,left=20truemm,right=20truemm]{geometry}


%%%%%%%%%%%%%%%%%%%%%%%%%%%%%%%% 皆さんが書き換えるのはここから %%%%%%%%%%%%%%%%%%%%%%%%%%%%%%%%%%%%%%%

\title{\TeX で自己紹介シートを作成しよう!}	% タイトル
\author{プログラム名 中村(友)研究室\\学籍番号 氏名} % 所属,学籍番号,名前
\date{}

%%%%%%%%%%%%%%% 練習7 %%%%%%%%%%%%%%%%%
% ここの数字をいじってページ数をちょうど2ページに合わせてみましょう.
\renewcommand{\baselinestretch}{1.0}
%%%%%%%%%%%%%%% ここまで %%%%%%%%%%%%%%%%%


\begin{document}
\maketitle

\renewcommand{\headrulewidth}{0.0pt}
\thispagestyle{fancy}
\lhead{}
\rhead{自己紹介 xxxx/xx/xx提出}
\cfoot{\thepage{}}

\input{mathdef}


\section{はじめに}
``main.tex''を編集しておおまかな``\TeX''の扱いが理解できたかと思います.
$2$章以降の指示に従い自己紹介シートを完成させGitHubで共有しましょう.

\section{事前準備}
まずは課題提出用のプライベートリポジトリを研究室のGitHubに作成しましょう.

\begin{enumerate}
  \item 研究室のGitHubに「(氏名)\_tex\_practice」(例:nakamura\_tex\_practice)という名前の{\bf プライベートリポジトリ}\footnote{パブリックリポジトリにすると,個人情報を全世界に発信してしまうので注意してください}を作成
  \item 作成したリポジトリを自身のPCにクローン
  \item TeXの課題ページからkadai.tex・mathdef.tex・bayes.pdfをダウンロードする
  \item ダウンロードしたkadai.texとmathdef.texをクローンしたリポジトリに入れる
  \item クローンしたリポジトリにfigというフォルダを作成し,bayes.pdfを作成したfigフォルダ内に入れる
  \item クローンしたリポジトリ内のkadai.texがコンパイルできることを確認(コンパイルできない場合は,コンパイルできるよう修正する)
  \item 追加したファイルをステージ(add)し,内容が分かりやすいコミットメッセージを付けてcommit・pushしましょう
\end{enumerate}


%%%%%%%%%%%%%%%%%%%%%%%%%%%%%%%%%%%%%%%
%「TeX使おう!」の3章参考
\section{名前$\cdot$誕生日$\cdot$出身}
\begin{description}
\item[指示1:]この章にsubsectionを作成し,名前・誕生日・出身地を記述しましょう.
\item[指示2:]それぞれのsubsectionにsubsubsectionを作成し,名前のsubsubsectionには呼ばれたことのあるニックネームを,誕生日のsubsubsectionにはその日が何の日か,出身地のsubsubsectionにはその土地の特産を記述しましょう.
\item[指示3:]PDFと変更したファイルをステージし,内容が分かりやすいコミットメッセージを付けてcommit・pushしましょう
\end{description}
%%%%%%%%%%%%%%%%%%%%%%%%%%%%%%%%%%%%%%%
%「TeX使おう!」の7章参考


\section{写真の挿入}
\begin{description}
\item[指示1:]どのような写真でもいいので,自身の写真を挿入してください(似顔絵や加工したものでもいいですが,なるべく顔が分かるようなものだとありがたいです
\footnote{事務所NGや宗教上の理由で挿入できない場合には,好きなものの画像でも構いません.})
\item[指示2:]インターネットから好きな写真をダウンロードして本文内に挿入し,画像の出典を参考文献に記載してください
\item[指示3:] その写真を簡単に説明する文章を加えて,その文章内で\verb|\label|,\verb|\ref|,\verb|\cite|を用いて,加えた画像と参考文献を参照してください
\item[指示4:]追加・変更したファイルをステージし,内容が分かりやすいコミットメッセージを付けてcommit・pushしましょう
\end{description}


%%%%%%%%%%%%%%%%%%%%%%%%%%%%%%%%%%%%%%%
%「TeX使おう!」の6章参考
\section{箇条書き}
\begin{description}
\item[指示1:]箇条書きで好きなものを10個以上列挙しましょう
\item[指示2:]変更したファイルをステージし,内容が分かりやすいコミットメッセージを付けてcommit・pushしましょう
\end{description}

%%%%%%%%%%%%%%%%%%%%%%%%%%%%%%%%%%%%%%%
%「TeX使おう!」の8章,5章参考
\section{表の挿入\&参考文献}
\begin{description}
\item[指示1:]以下の設問を1列目に,回答を2列目に記述した表($8 \times 2$)を作成しましょう(ない場合は,``無し''でもいいです)
\item[指示2:]回答のURL等ある場合は,``footnote''を使いましょう.
%ヒント!!!
%\begin{table}~\end{table}までを,\begin{savenotes}と\end{savenotes}で挟みましょう.
\item[指示3:]回答に関する参考文献を1つ以上追加し,表内で参照しましょう
\item[指示4:]変更したファイルをステージし,内容が分かりやすいコミットメッセージを付けてcommit・pushしましょう
\end{description}

\noindent {\bf<設問>}
\begin{enumerate}
\setlength{\parskip}{-0.2mm}
\setlength{\itemsep}{-0.1mm}
  \item 血液型
  \item 好きな曲
  \item 好きなゲーム
  \item 好きなアニメ$\cdot$ドラマ
  \item サークル$\cdot$部活(大学以前でもいいです)
  \item オススメのご飯$\cdot$ご飯屋
  \item オススメのお菓子
  \item オススメの観光地
\end{enumerate}

%%%%%%%%%%%%%%%%%%%%%%%%%%%%%%%%%%%%%%%
%「TeX使おう!」の9章参考
\begin{figure}[t]
 \centering 
  \includegraphics[scale=0.8]{fig/bayes.pdf}
  \caption{ベイズの定理}
  \label{fig:bayes}
\end{figure}

\section{数式}
\begin{description}
\item[指示1:]図\ref{fig:bayes}を数式で作成してください.
\item[指示2:]変更したファイルをステージし,内容が分かりやすいコミットメッセージを付けてcommit・pushしましょう
\end{description}

%%%%%%%%%%%%%%%%%%%%%%%%%%%%%%%%%%%%%%%
%「TeX使おう!」の10章参考
\section{ページ数の調整}
\begin{description}
\item[指示1:]ソースファイルの上部に書かれている\verb|\renewcommand{\baselinestretch}{1.0}|を用いて, 2ページちょうどに調整してみましょう.
\item[指示2:]変更したファイルをステージし,内容が分かりやすいコミットメッセージを付けてcommit・pushしましょう
\end{description}

%%%%%%%%%%%%%%%%%%%%%%%%%%%%%%%%%%%%%%%
\section{提出方法}
指定された期日までに本自己紹介シートを作成して,GitHubへプッシュすることで提出がされたものとみまします.
リポジトリ内に,コンパイルに必要な全てのファイルとコンパイルされたPDFが入っていることを必ず確認してください.
最後にリポジトリを新たにクローンして,コンパイルできるかを確認するといかと思います.
%%%%%%%%%%%%%%%%%%%%%%%%%%%%%%%%%%%%%%%

\begin{thebibliography}{99}

\bibitem{nakamura2014}
中村友昭,
``\TeX を使おう!'', 
ゼミ資料, pp.1-5, 2014

\end{thebibliography}

\thispagestyle{fancy}
\renewcommand{\headrulewidth}{0.0pt}
\rhead{}
%指導教員の数で以下のコメントアウトを切り替える
%\rfoot{\underline{指導教員氏名\hspace{8em}印}}
%\cfoot{\underline{主任指導教員氏名\hspace{7em}印} ~~ \underline{指導教員氏名\hspace{7em}印} ~~  \underline{指導教員氏名\hspace{7em}印}}
\end{document} 